\documentclass[a4paper,12pt,twoside]{report}

\usepackage[english]{babel}
\usepackage[utf8]{inputenc}

%------------------------------
% Header settings for all pages
%------------------------------
\usepackage{fancyhdr}

\pagestyle{fancy}
\fancyhf{}
\rhead{Kian Nejadfard}
\lhead{Research Kernel}
%------------------------------

%------------------------------
% For timeline
%------------------------------
\newcommand{\foo}{\hspace{-2.3pt}$\bullet$ \hspace{5pt}}
\usepackage{graphicx}
%------------------------------

%------------------------------
% Bibliography configurations
%------------------------------
\usepackage[backend=biber, sorting=none]{biblatex}
\bibliography{notes.bib}
%------------------------------

%------------------------------
% Customizations for lstlisting package
%------------------------------
\usepackage{listings}
\usepackage[utf8]{inputenc}

\usepackage{xcolor}

\definecolor{codegreen}{rgb}{0,0.6,0}
\definecolor{codegray}{rgb}{0.5,0.5,0.5}
\definecolor{codepurple}{rgb}{0.58,0,0.82}
\definecolor{backcolour}{rgb}{0.95,0.95,0.92}

\lstdefinestyle{mystyle}{
	backgroundcolor=\color{backcolour},
	commentstyle=\color{codegreen},
	keywordstyle=\color{magenta},
	numberstyle=\tiny\color{codegray},
	stringstyle=\color{codepurple},
	basicstyle=\ttfamily\footnotesize,
	breakatwhitespace=false,
	breaklines=true,
	captionpos=b,
	keepspaces=true,
	numbers=left,
	numbersep=5pt,
	showspaces=false,
	showstringspaces=false,
	showtabs=false,
	tabsize=2
}

\lstset{style=mystyle}
%------------------------------

\usepackage{fancyvrb}
%------------------------------

%------------------------------
% Table Configurations
%------------------------------
\usepackage{multirow}
% Spacing between cell content and left/right borders
\setlength{\tabcolsep}{5pt}
% The height of each row is set to 1.5 relative to its default height.
\renewcommand{\arraystretch}{1.3}

% This is to be able to have tables stay within the section they are written at.
\usepackage{float}

\usepackage{longtable}
%------------------------------

\title{Project RKern}
\author{Kian Nejadfard}

\begin{document}
    \maketitle

    \begin{abstract}
    	RKern is my new research project for developing a kernel from scratch in C++ to evaluate how the use of zero-overhead abstraction mechanisms can affect a kernel project in terms of codebase maintainability, modularity, usability, and performance.
    \end{abstract}

	\chapter{Preface}
        \input{"texfiles/preface.tex"}

    \chapter{Prerequisites}
        \input{"texfiles/prerequisites.tex"}

    \chapter{General Topics}
        \section{Kernels, What They Are, And How They Differ}
            \section{What Is a Kernel?}
TODO

\section{Kernel Types}
TODO

\section{A Brief History}
TODO: review how Unix, Linux, BSDs, etc. were started. Including a timeline, and also review the timeline of Assembly, C, and C++. The idea is to find out if the main reason for today's kernels' use of C is just the time that they started development and the fact that C was the best and only sane choice at the time. Clearly, after decades of development, you can't just change languages of the kernel. So... maybe it is time for a fresh start!

        \section{Binary File Analysis}
            \section{Disassembling The Binary}

\begin{verbatim}
llvm-objdump --disassemble-all rkern.bin
objdump -d rkern.bin
\end{verbatim}

        \section{Linker}
            \subsection{What Do Linkers Do?}
TODO: use reference http://www.bravegnu.org/gnu-eprog/linker.html

\subsection{Linker Script}
A linker script can define 4 pieces of information:
\begin{enumerate}
    \item \textbf{Memory layout}
    \item \textbf{Section definitions} - Defines the structure of the binary file that will be produced by the linker program.
    \item \textbf{Options} - Specifications of architecture, entry point, etc. if needed.
    \item \textbf{Symbols} - Variables that have to be injected into the program at link time.
\end{enumerate}\cite{memfaultLinkerScripts}

\subsubsection{Memory Layout}
In order to allocate program space, the linker needs to know how much memory is available, and at what addresses that memory exists. This is what the \verb|MEMORY| definition in the linker script is for.

The syntax for \verb|MEMORY| is as follows:
\begin{verbatim}
MEMORY
{
    name [(attr)] : ORIGIN = origin, LENGTH = len
    ...
}
\end{verbatim}
Where:
\begin{itemize}
    \item \verb|name| is the region's name. The choice of name is arbitrary as they do not carry any specific meaning. Typical names include \textbf{flash} and \textbf{ram}.
    \item \verb|(attr)| are optional attributes for the region, such as \verb|w| (writable), \verb|r| (readable), \verb|x| (executable). Flash memory is usually \verb|rx| while ram is usually \verb|wrx|. Notice that these attributes do not actually set memory, rather they just describe the properties of the memory region.
    \item \verb|origin| is the start address of the memory region.
    \item \verb|len| is the size of the memory region in bytes.
\end{itemize}

\subsubsection{Program Headers}
Also known as \textit{segments}, the \textit{program headers} describe how a program is loaded in memory from an ELF object file format. While the linker creates reasonable program headers by default, sometimes it may be necessary to customize them.\cite{gnuldProgramHeaders}

In order to observe the program headers of an ELF file, the following command may be used:
\begin{verbatim}
    objdump -p <elf file>
\end{verbatim}

Program headers may be defined by using the \lstinline|PHDRS| command in the linker script:
\begin{verbatim}
PHDRS
{
    name type [ FILEHDR ] [ PHDRS ] [ AT ( address ) ]
    [ FLAGS ( flags ) ] ;
}
\end{verbatim}

Certain program header types describe segments of memory which are loaded from the ELF file by the system loader. In the linker script, the contents of these segments are specified by directing allocated output sections to be placed in the segment. To do this, the command describing the output section in the \lstinline|SECTIONS| command should use \lstinline|:name|, where \lstinline|name| is the program header name as it appears in the \lstinline|PHDRS| command.\cite{gnuldProgramHeaders}

If a section is placed in one or more segments using `:name', then all subsequent allocated sections which do not specify `:name' are placed in the same segments.\cite{gnuldProgramHeaders}

The \lstinline|FILEHDR| and \lstinline|PHDRS| keywords which may appear after the program header type also indicate contents of the segment of memory. The \lstinline|FILEHDR| keyword means that the segment should include the ELF file header. The \lstinline|PHDRS| keyword means that the segment should include the ELF program headers themselves.\cite{gnuldProgramHeaders}

\lstinline|type| may be one of the following\cite{gnuldProgramHeaders}:
\begin{itemize}
    \item \lstinline|PT_NULL| - Indicates an unused program header.
    \item \lstinline|PT_LOAD| - Indicates that this program header describes a segment to be loaded from the file.
    \item \lstinline|PT_DYNAMIC| - Indicates a segment where dynamic linking information can be found.
    \item \lstinline|PT_INTERP| - Indicates a segment where the name of the program interpreter may be found.
    \item \lstinline|PT_NOTE| - Indicates a segment holding note information.
    \item \lstinline|PT_SHLIB| - A reserved program header type, defined but not specified by the ELF ABI.
    \item \lstinline|PT_PHDR| - Indicates a segment where the program headers may be found.
    \item \lstinline|expression| - An expression giving the numeric type of the program header. This may be used for types not defined above.
\end{itemize}

It is possible to specify that a segment should be loaded at a particular address in memory. This is done using an \lstinline|AT| expression. This is identical to the \lstinline|AT| command used in the \lstinline|SECTIONS| command. Using the \lstinline|AT| command for a program header overrides any information in the \lstinline|SECTIONS| command.\cite{gnuldProgramHeaders}

Knowing the above, the following is a simple program header definition for use with HiFive Rev B development board:
\begin{lstlisting}
PHDRS
{
    flash PT_LOAD;
    ram PT_NULL;
}
\end{lstlisting}

\subsubsection{Sections}
The \lstinline|SECTIONS| command controls exactly where input sections are placed into output sections, their order in the output file, and to which output sections they are allocated.\cite{gnuldOutputSections}

If you do not use a SECTIONS command, the linker places each input section into an identically named output section in the order that the sections are first encountered in the input files. If all input sections are present in the first file, for example, the order of sections in the output file will match the order in the first input file.\cite{gnuldOutputSections}

In a section definition, you can specify the contents of an output section by listing particular input files, by listing particular input-file sections, or by a combination of the two. You can also place arbitrary data in the section, and define symbols relative to the beginning of the section.\cite{gnuldOutputSections}
\begin{verbatim}
SECTIONS {
    secname : {
        filename( section , section, ... )
    }
}
\end{verbatim}
The whitespace around \lstinline|secname| is required, so that the section name is unambiguous. The other whitespace shown is optional. You do need the colon \lstinline|:| and the braces \lstinline|{}|, however.\cite{gnuldOutputSections}

The linker will not create output sections which do not have any contents.\cite{gnuldOutputSections}

The \lstinline|*( COMMON )| notation can be used to specify where uninitialized data should be placed in the output file.\cite{gnuldOutputSections}

\subsubsection{Options}

\subsubsection{Symbols}

        \section{Communication Protocols}
            This section is dedicated to exploring various mechanisms and communication protocols used in embedded devices for working with peripheral devices.

\subsection{GPIO}
General-Purpose Input/Output, abbreviated as GPIO, is one of the most basic means of communication between a micro-controller and a peripheral device that is connected to it.

General-purpose Input/Output pins have no predetermined usage. As such, various aspects of a GPIO pin may be configured:\cite{giometti2017gnu}
\begin{itemize}
    \item Enabled or disabled
    \item Used as input or output
    \item Read or write output value
    \item Read input value
    \item Set value as pulled-up or pulled-down by default
\end{itemize}

In fact, the GPIO pins are generic enough that they may be used to emulate another digital interface controller. Other than emulating, GPIO pins may be used for signal control (reset, power enable, suspend), managing relays, LEDS, switches, and buttons.\cite{giometti2017gnu}

\subsection{UART}

\subsection{I2C}

\subsection{SPI}

    \chapter{HiFive1 Rev B}
        After learning about what linkers do and what a linker script is composed of, we should consult the documentation of the hardware that we want to work with, in order to figure out critical information that we need to use for the compilation and linking process.

        \section{Gathering Hardware Information}
            \input{"texfiles/hifive_revb/gathering_hardware_information.tex"}

        \section{Memory Map}
           	\input{"texfiles/hifive_revb/memory_map.tex"}

        \section{GPIO}
            The following tables show the available GPIO instance and its memory map for HiFive1 Rev B board:
\begin{table}[H]
    \centering
    \begin{tabular}{| p{3cm} | c | c | p{3cm} |}
        \hline
        \textbf{Instance Number} & \textbf{Address} & \textbf{Pins}\\
        \hline
        \hline
        0 & 0x10012000 & 32\\
        \hline
    \end{tabular}
    \caption{GPIO Instances}
\end{table}

\begin{table}[H]
    \centering
    \begin{tabular}{| p{3cm} | c | c | p{3cm} |}
        \hline
        \textbf{Offset} & \textbf{Name} & \textbf{Description}\\
        \hline
        \hline
        0x00 & \lstinline|input_val| & Pin input value\\
        0x04 & \lstinline|input_en| & Pin input enable*\\
        0x08 & \lstinline|output_en| & Pin output enable*\\
        0x0C & \lstinline|output_val| & Pin output value\\
        0x10 & \lstinline|pue| & Internal pull-up enable*\\
        0x14 & \lstinline|ds| & Pin drive strength\\
        0x18 & \lstinline|rise_ie| & Rise interrupt enable\\
        0x1C & \lstinline|rise_ip| & Rise interrupt pending\\
        0x20 & \lstinline|fall_ie| & Fall interrupt enable\\
        0x24 & \lstinline|fall_ip| & Fall interrupt pending\\
        0x28 & \lstinline|high_ie| & High interrupt enable\\
        0x2C & \lstinline|high_ip| & High interrupt pending\\
        0x30 & \lstinline|low_ie| & Low interrupt enable\\
        0x34 & \lstinline|low_ip| & Low interrupt pending\\
        0x38 & \lstinline|iof_en| & IO function enable (missing in reference doc)\\
        0x40 & \lstinline|out_xor| & Output XOR (invert)\\
        \hline
    \end{tabular}
    \caption{GPIO Memory Map}
    \label{tab:gpio_memory_map}
\end{table}

Per FE310-G002's manual\cite{fe310g002man}, only naturally aligned 32-bit memory accesses are supported. Moreover, registers marked with an asterisk in table \ref{tab:gpio_memory_map} are asynchronously reset to 0. Registers without an asterisk are synchronously reset to 0.

\subsection{On-board LEDs}
HiFive1 Rev B has three LEDs on the board, which may be accessed via the following GPIO pins:
\begin{itemize}
    \item LED 0 - GPIO 22
    \item LED 1 - GPIO 19
    \item LED 2 - GPIO 21
\end{itemize}

        \section{Flashing With OpenOCD}
            \input{texfiles/hifive_revb/flashing_with_openocd.tex}

        \section{Attaching Terminal}
            \input{texfiles/hifive_revb/attaching_terminal.tex}

        \section{Debugging With GDB and Qemu}
\begin{lstlisting}
qemu-system-riscv32 -M sifive_e,revb=true -s -S -kernel rkern.bin
riscv32-unknown-elf-gdb rkern.bin
\end{lstlisting}

Once in gdb:
\begin{lstlisting}
target remote localhost:1234
\end{lstlisting}

    \printbibliography
\end{document}
