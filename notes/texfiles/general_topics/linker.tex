\subsection{What Do Linkers Do?}
TODO: use reference http://www.bravegnu.org/gnu-eprog/linker.html

\subsection{Linker Script}
A linker script can define 4 pieces of information:
\begin{enumerate}
    \item \textbf{Memory layout}
    \item \textbf{Section definitions} - Defines the structure of the binary file that will be produced by the linker program.
    \item \textbf{Options} - Specifications of architecture, entry point, etc. if needed.
    \item \textbf{Symbols} - Variables that have to be injected into the program at link time.
\end{enumerate}\cite{memfaultLinkerScripts}

\subsubsection{Memory Layout}
In order to allocate program space, the linker needs to know how much memory is available, and at what addresses that memory exists. This is what the \verb|MEMORY| definition in the linker script is for.

The syntax for \verb|MEMORY| is as follows:
\begin{verbatim}
MEMORY
{
    name [(attr)] : ORIGIN = origin, LENGTH = len
    ...
}
\end{verbatim}
Where:
\begin{itemize}
    \item \verb|name| is the region's name. The choice of name is arbitrary as they do not carry any specific meaning. Typical names include \textbf{flash} and \textbf{ram}.
    \item \verb|(attr)| are optional attributes for the region, such as \verb|w| (writable), \verb|r| (readable), \verb|x| (executable). Flash memory is usually \verb|rx| while ram is usually \verb|wrx|. Notice that these attributes do not actually set memory, rather they just describe the properties of the memory region.
    \item \verb|origin| is the start address of the memory region.
    \item \verb|len| is the size of the memory region in bytes.
\end{itemize}

\subsubsection{Program Headers}
Also known as \textit{segments}, the \textit{program headers} describe how a program is loaded in memory from an ELF object file format. While the linker creates reasonable program headers by default, sometimes it may be necessary to customize them.\cite{gnuldProgramHeaders}

In order to observe the program headers of an ELF file, the following command may be used:
\begin{verbatim}
    objdump -p <elf file>
\end{verbatim}

Program headers may be defined by using the \lstinline|PHDRS| command in the linker script:
\begin{verbatim}
PHDRS
{
    name type [ FILEHDR ] [ PHDRS ] [ AT ( address ) ]
    [ FLAGS ( flags ) ] ;
}
\end{verbatim}

Certain program header types describe segments of memory which are loaded from the ELF file by the system loader. In the linker script, the contents of these segments are specified by directing allocated output sections to be placed in the segment. To do this, the command describing the output section in the \lstinline|SECTIONS| command should use \lstinline|:name|, where \lstinline|name| is the program header name as it appears in the \lstinline|PHDRS| command.\cite{gnuldProgramHeaders}

If a section is placed in one or more segments using `:name', then all subsequent allocated sections which do not specify `:name' are placed in the same segments.\cite{gnuldProgramHeaders}

The \lstinline|FILEHDR| and \lstinline|PHDRS| keywords which may appear after the program header type also indicate contents of the segment of memory. The \lstinline|FILEHDR| keyword means that the segment should include the ELF file header. The \lstinline|PHDRS| keyword means that the segment should include the ELF program headers themselves.\cite{gnuldProgramHeaders}

\lstinline|type| may be one of the following\cite{gnuldProgramHeaders}:
\begin{itemize}
    \item \lstinline|PT_NULL| - Indicates an unused program header.
    \item \lstinline|PT_LOAD| - Indicates that this program header describes a segment to be loaded from the file.
    \item \lstinline|PT_DYNAMIC| - Indicates a segment where dynamic linking information can be found.
    \item \lstinline|PT_INTERP| - Indicates a segment where the name of the program interpreter may be found.
    \item \lstinline|PT_NOTE| - Indicates a segment holding note information.
    \item \lstinline|PT_SHLIB| - A reserved program header type, defined but not specified by the ELF ABI.
    \item \lstinline|PT_PHDR| - Indicates a segment where the program headers may be found.
    \item \lstinline|expression| - An expression giving the numeric type of the program header. This may be used for types not defined above.
\end{itemize}

It is possible to specify that a segment should be loaded at a particular address in memory. This is done using an \lstinline|AT| expression. This is identical to the \lstinline|AT| command used in the \lstinline|SECTIONS| command. Using the \lstinline|AT| command for a program header overrides any information in the \lstinline|SECTIONS| command.\cite{gnuldProgramHeaders}

Knowing the above, the following is a simple program header definition for use with HiFive Rev B development board:
\begin{lstlisting}
PHDRS
{
    flash PT_LOAD;
    ram PT_NULL;
}
\end{lstlisting}

\subsubsection{Sections}
The \lstinline|SECTIONS| command controls exactly where input sections are placed into output sections, their order in the output file, and to which output sections they are allocated.\cite{gnuldOutputSections}

If you do not use a SECTIONS command, the linker places each input section into an identically named output section in the order that the sections are first encountered in the input files. If all input sections are present in the first file, for example, the order of sections in the output file will match the order in the first input file.\cite{gnuldOutputSections}

In a section definition, you can specify the contents of an output section by listing particular input files, by listing particular input-file sections, or by a combination of the two. You can also place arbitrary data in the section, and define symbols relative to the beginning of the section.\cite{gnuldOutputSections}
\begin{verbatim}
SECTIONS {
    secname : {
        filename( section , section, ... )
    }
}
\end{verbatim}
The whitespace around \lstinline|secname| is required, so that the section name is unambiguous. The other whitespace shown is optional. You do need the colon \lstinline|:| and the braces \lstinline|{}|, however.\cite{gnuldOutputSections}

The linker will not create output sections which do not have any contents.\cite{gnuldOutputSections}

The \lstinline|*( COMMON )| notation can be used to specify where uninitialized data should be placed in the output file.\cite{gnuldOutputSections}

\subsubsection{Options}

\subsubsection{Symbols}