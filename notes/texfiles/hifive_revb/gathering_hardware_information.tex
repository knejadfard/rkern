After reviewing the documentations of the \textit{HiFive1 Rev B} board, as well as the \textit{FE310-G002} core that comes with it, I have discovered the following information:
\begin{itemize}
    \item The FE310-G002 core is configured to support the RV32IMAC ISA options. This specifies the architecture to be used when producing the binary file.\cite{hifive1RevBConfig}

    \item The data SRAM is 16 KiB.\cite{hifive1RevBConfig}

    \item The system mask ROM is 8 KiB in size and contains simple boot code.\cite{hifive1RevBConfig}

    \item The mask ROM (MROM) is fixed at design time, and is located on the peripheral bus on FE310-G002, but instructions fetched from MROM are cached by the core’s I-cache. The MROM contains an instruction at address 0x10000 which jumps to the OTP start address at 0x20000.\cite{fe310g002manBootProcess}

    \item A dedicated quad-SPI (QSPI) flash interface is provided to hold code and data for the system. The QSPI interface supports burst reads of 32 bytes over TileLink to accelerate instruction cache refills. The QSPI can be programmed to support eXecute-In-Place modes to reduce SPI command overhead on instruction cache refills. The QSPI interface also supports single-word data reads over the primary TileLink interface, as well as programming operations using memory-mapped control registers.\cite{hifive1RevBConfig}

    \item FE310-G002 boots by jumping to the beginning of the OTP memory and executing the code found there. As shipped, the OTP memory at the boot location is programmed to jump immediately to the end of the OTP memory, which in turn jumps to the beginning of the SPI Flash at \textbf{0x20000000}.\cite{hifive1RevBBootCode}

    \item The OTP is located on the peripheral bus, with both a control register interface to program the OTP, and a memory read port interface to fetch words from the OTP. Instruction fetches from the OTP memory read port are cached in the E31 core’s instruction cache.\cite{fe310g002manBootProcess}

    \item The HiFive1 Rev B Board is shipped with a modifiable boot loader at the beginning of SPI Flash (0x20000000). At the end of this program’s execution the core jumps to the main user portion of code at \textbf{0x20010000}. This program is designed to allow quick boot, but also a safe reboot option if a “bad” program is flashed into the SPI Flash. A bad program is one which makes it impossible for the programmer to communicate with the board. For example, a program which disables FE310’s active clock, or which puts the core to sleep with no way of waking it up. Bad programs can always be restarted using the RESET button, and using the “safe” bootloader can be halted before they perform any unsafe behavior.\cite{hifive1RevBBootLoader}

    \item To activate normal boot mode, press the RESET button on the HiFive1 Rev B. After approximately 1 second, the green LED will flash for 1/2 second, then the user program will execute.\cite{hifive1RevBBootLoader}

    \item To activate safe boot mode, press the RESET button. When the green LED flashes, immediately press the RESET button again. After 1 second, the red LED will blink. The user program will not execute, and the programmer can connect. To exit “safe” boot mode, press the RESET button a final time.\cite{hifive1RevBBootLoader}

    \item There are 3 serial peripheral interface (SPI) controllers. Each controller provides a means for serial communication between the FE310-G002 and off-chip devices, like quad-SPI Flash memory. Each controller supports master-only operation over single-lane, dual-lane, and quad-lane protocols. Each controller supports burst reads of 32 bytes over TileLink to accelerate instruction cache refills. 1 SPI controller can be programmed to support eXecute-In-Place (XIP) modes to reduce SPI command overhead on instruction cache refills.\cite{fe310g002manOverview}

    \item Two universal asynchronous receiver/transmitter (UARTs) are available and provide a means for serial communication between the FE310-G002 and off-chip devices.\cite{fe310g002manOverview}

    \item The FE310-G002 has an I2C controller to communicate with external I2C devices, such as sensors, ADCs, etc.\cite{fe310g002manOverview}

    \item The FE310-G002 provides external debugger support over an industry-standard JTAG port, including 8 hardware-programmable breakpoints per hart.\cite{fe310g002manOverview}
\end{itemize}