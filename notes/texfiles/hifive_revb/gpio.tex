The following tables show the available GPIO instance and its memory map for HiFive1 Rev B board:
\begin{table}[H]
    \centering
    \begin{tabular}{| p{3cm} | c | c | p{3cm} |}
        \hline
        \textbf{Instance Number} & \textbf{Address} & \textbf{Pins}\\
        \hline
        \hline
        0 & 0x10012000 & 32\\
        \hline
    \end{tabular}
    \caption{GPIO Instances}
\end{table}

\begin{table}[H]
    \centering
    \begin{tabular}{| p{3cm} | c | c | p{3cm} |}
        \hline
        \textbf{Offset} & \textbf{Name} & \textbf{Description}\\
        \hline
        \hline
        0x00 & \lstinline|input_val| & Pin input value\\
        0x04 & \lstinline|input_en| & Pin input enable*\\
        0x08 & \lstinline|output_en| & Pin output enable*\\
        0x0C & \lstinline|output_val| & Pin output value\\
        0x10 & \lstinline|pue| & Internal pull-up enable*\\
        0x14 & \lstinline|ds| & Pin drive strength\\
        0x18 & \lstinline|rise_ie| & Rise interrupt enable\\
        0x1C & \lstinline|rise_ip| & Rise interrupt pending\\
        0x20 & \lstinline|fall_ie| & Fall interrupt enable\\
        0x24 & \lstinline|fall_ip| & Fall interrupt pending\\
        0x28 & \lstinline|high_ie| & High interrupt enable\\
        0x2C & \lstinline|high_ip| & High interrupt pending\\
        0x30 & \lstinline|low_ie| & Low interrupt enable\\
        0x34 & \lstinline|low_ip| & Low interrupt pending\\
        0x38 & \lstinline|iof_en| & IO function enable (missing in reference doc)\\
        0x40 & \lstinline|out_xor| & Output XOR (invert)\\
        \hline
    \end{tabular}
    \caption{GPIO Memory Map}
    \label{tab:gpio_memory_map}
\end{table}

Per FE310-G002's manual\cite{fe310g002man}, only naturally aligned 32-bit memory accesses are supported. Moreover, registers marked with an asterisk in table \ref{tab:gpio_memory_map} are asynchronously reset to 0. Registers without an asterisk are synchronously reset to 0.

\subsection{On-board LEDs}
HiFive1 Rev B has three LEDs on the board, which may be accessed via the following GPIO pins:
\begin{itemize}
    \item LED 0 - GPIO 22
    \item LED 1 - GPIO 19
    \item LED 2 - GPIO 21
\end{itemize}